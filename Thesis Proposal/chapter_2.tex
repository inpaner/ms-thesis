%%%%%%%%%%%%%%%%%%%%%%%%%%%%%%%%%%%%%%%%%%%%%%%%%%%%%%%%%%%%%%%%%%%%%%%%%%%%%%%%%%%%%%%%%%%%%%%%%%%%%%
%
%   Filename    : chapter_2.tex 
%
%   Description : This file will contain your Review of Related Literature.
%                 
%%%%%%%%%%%%%%%%%%%%%%%%%%%%%%%%%%%%%%%%%%%%%%%%%%%%%%%%%%%%%%%%%%%%%%%%%%%%%%%%%%%%%%%%%%%%%%%%%%%%%%

\chapter{Review of Related Literature}
\label{sec:relatedlit}

This chapter discusses the features, capabilities, and limitations of existing research, algorithms, or software 
that are related/similar to the thesis.

 The reviewed works and software must be arranged either in chronological order, or by area (from general to specific).  
Observe a consistent format when presenting each of the reviewed works. This must be selected in consultation with the prospective adviser.

\textcolor{red}{DO NOT FORGET to cite your references.}


\begin{comment}
%
% IPR acknowledgement: the contents withis this comment are from Ethel Ong's slides on RRL.
%
Guide on Writing your RRL chapter
 
1. Identify the keywords with respect to your research
      One keyword = One document section
                Examples: 2.1 Story Generation Systems
			 2.2 Knowledge Representation

2.  Find references using these keywords

3.  For each of the references that you find,
        Check: Is it relevant to your research?
        Use their references to find more relevant works.

4. Identify a set of criteria for comparison.
       It will serve as a guide to help you focus on what to look for

5. Write a summary focusing on -
       What: A short description of the work
       How: A summary of the approach it utilized
       Findings: If applicable, provide the results
        Why: Relevance to your work

6. At the end of each section,  show a Table of Comparison of the related works 
   and your proposed project/system

\end{comment}

\section{Review of Related Paper}
This section contains a review of research papers that:
%
% IPR acknowledgement: the following list of items are from Ethel Ong's slides RRL.
%
\begin{itemize}
\item Describes work on a research area that is similar or relevant to yours
\item Describes work on a domain that is similar or relevant to yours
\item Uses an algorithm that may be useful to your work
\item Uses a software / tool that may be useful to your work
\end{itemize}

\section{Review of Related Software}
This section contains a review of software systems that:
%
% IPR acknowledgement: the following list of items are from Ethel Ong's slides on RRL.
%
\begin{itemize}
   \item Belongs to a research area similar to yours
   \item Addresses a need or domain similar to yours
   \item Is your predecessor
\end{itemize}


\section{Medical Diagnosis}
Medical diagnosis is the process of determining which disease or condition explains a person's symptoms and signs. 
The information required to reach a diagnosis is usually collected from the history of present illness and physical examination.

\section{Twenty Questions}
Twenty questions is a spoken parlor game which encourages deductive reasoning and creativity \cite{wiki:twenty_questions}.
Traditionally, one player -- the answerer -- gets assigned an object which is not revealed.
Another person or group -- the questioners -- asks the answerer questions which can be answered with either "Yes" or "No". 
Lying is not allowed.
The game ends when the questioners correctly guess the answerer's object or when twenty questions have been asked without a correct guess.

Several programs were developed that attempt to simulate the questioner. 
The user thinks of an answer and the program would ask questions which aims to identify the answer using the least number of questions. The user would then respond with either "Yes" or "No".

One popular program is Akinator, which is freely accessible online \cite{Akinator:2016}.
Akinator's database is limited to characters both fictional and non-fictional, but this restriction enables it to determine answers with high accuracy.
The system is dynamic and learns from previous games, further increasing its accuracy.
As of writing, the system has played 226 million games with at least 60 thousand games daily.
Akinator's dynamicity also enables it to learn new characters as they get introduced in real life.
In this manner, Akinator can guess characters from pop culture and current events.


