%%%%%%%%%%%%%%%%%%%%%%%%%%%%%%%%%%%%%%%%%%%%%%%%%%%%%%%%%%%%%%%%%%%%%%%%%%%%%%%%%%%%%%%%%%%%%%%%%%%%%%
%
%   Filename    : chapter_2.tex 
%
%   Description : This file will contain your Review of Related Literature.
%                 
%%%%%%%%%%%%%%%%%%%%%%%%%%%%%%%%%%%%%%%%%%%%%%%%%%%%%%%%%%%%%%%%%%%%%%%%%%%%%%%%%%%%%%%%%%%%%%%%%%%%%%

\chapter{Review of Related Literature}
\label{sec:relatedlit}

This chapter discusses the features, capabilities, and limitations of existing research, algorithms, or software 
that are related/similar to the thesis.

\section{Medical Diagnosis}
Medical diagnosis is the process of determining which disease or condition explains a person's symptoms and signs. 
The information required to reach a diagnosis is usually collected from the history of present illness and physical examination.

\section{Epidemiology of Diseases}
Epidemiology is the branch of medicine that studies the distribution and determinants of healt-related states or events, and the application of this study to the control of diseases and other health problems \footnote{http://www.who.int/topics/epidemiology/en/}.
Different methods can be used to carry out epidemiological research, including surveillance, descriptive studies, and analytical studies.

In terms of statistics, diseases can be measured using different metrics \cite{nydoh:1999}. 
\begin{itemize}
\item Incidence determines a person's probability of being diagnosed with a disease over a given period of time.
\item Prevalence on the other hand determines a person's likelihood of having a disease.
\end{itemize}




\section{Twenty Questions}
Twenty questions is a spoken parlor game which encourages deductive reasoning and creativity \cite{wiki:twenty_questions}.
Traditionally, one player -- the answerer -- gets assigned an object which is not revealed.
Another person or group -- the questioners -- asks the answerer questions which can be answered with either Yes or No. 
Lying is not allowed.
The game ends when the questioners correctly guess the answerer's object or when twenty questions have been asked without a correct guess.

Several programs were developed that attempt to simulate the questioner. 
The user thinks of an answer and the program would ask questions which aims to identify the answer using the least number of questions. The user would then respond with either "Yes" or "No".

One such program is 20Q, which is freely accessible online \cite{20q:2016}. 
Unlike traditional rules where only Yes or No can be answered, 20Q accepts different degrees of answer of yes and no.
These include Usually which is described as most of the time the answer would be Yes, and Irrelevant where the question asked by 20Q does not apply to the object.
As of writing, the system has played 87 million games online.

Another popular web program is Akinator \cite{Akinator:2016}.
Akinator's database is limited to characters both fictional and non-fictional, but this restriction enables it to determine answers with high accuracy.
The system is dynamic and learns from previous games, further increasing its accuracy.
As of writing, the system has played 226 million games online with at least 60 thousand games daily.
Akinator's dynamicity also enables it to learn new characters as they get introduced in real life.
In this manner, Akinator can guess characters from pop culture and current events.
