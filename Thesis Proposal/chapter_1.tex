%%%%%%%%%%%%%%%%%%%%%%%%%%%%%%%%%%%%%%%%%%%%%%%%%%%%%%%%%%%%%%%%%%%%%%%%%%%%%%%%%%%%%%%%%%%%%%%%%%%%%%
%
%   Filename    : chapter_1.tex 
%
%   Description : This file will contain your Research Description.
%                 
%%%%%%%%%%%%%%%%%%%%%%%%%%%%%%%%%%%%%%%%%%%%%%%%%%%%%%%%%%%%%%%%%%%%%%%%%%%%%%%%%%%%%%%%%%%%%%%%%%%%%%

\chapter{Research Description}
\label{sec:researchdesc}    %--note: labels help you with hyperlink editing (using your IDE)
This chapter discusses the current technologies in expert systems and diagnostic systems.
This also covers the objectives, scope and limitations of the research, significance of the research, and the research methodology.

\section{Overview of the Current State of Technology}
\label{sec:overview}
The proportion of Filipinos dying without medical attention has risen to seventy percent.
This is further complicated by Filipino doctors leaving to work overseas.
Since 2000, more than six thousand doctors-turned-nurses have left the country with five thousand more expected to go \cite{Harden:2008}.
While the Philippines does not yet have a shortage of doctors, there is a shortage in poor rural areas.
It is estimated that one doctor serves a population of thirty thousand, a stark contrast to the ideal ratio of one doctor for every thousand \cite{Manongdo:2014}.
Overall, an estimated 10.8 million families rely on rural and city health units as well as barangay health stations for their primary care \cite{gmc:2013}.

Expert systems are computer applications which employ non-algorithmic expertise for solving certain types of problems.
 These include diagnostic applications, financial planning, and real-time monitoring \cite{Merritt:2010}.
A basic implementation of an expert system makes use of several components:
a knowledge base which form the human expert's knowledge; 
the inference engine which defines the manner in which new knowledge can be derived from the current knowledge; 
working memory which stores the information concerning the specific problem to be solved; 
and an interface with which the user communicates with the system \cite{Agarwal:2014}. 

Expert systems are usually divided into two categories: goal driven reasoning and data driven reasoning \cite{Merritt:2010}. 
Goal driven reasoning or backward chaining employs rules to repetitively break a goal into smaller sub-goals which are easier to prove. 
Identification and diagnostic systems fit this model, with the aim of determining the correct diagnosis. 
Data driven reasoning or forward chaining is used in configuration problems where inputs vary and can be combined in an almost infinite number of ways. 
The system keeps track of the current state of the problem and identifies rules which will move the state closer to a final solution .

\citeA{Dayupay:2015} developed a system that automatically generates the history of present illness (HPI) of a patient.
The HPI system makes use of an expert system to guide the user in methodically asking questions regarding their chief complaint. 
The expert system is initially data-driven then switches to being goal-driven once sufficient symptoms have been identified.
However, this procedure is highly linear; the expert system would not switch back to a data-driven mode.
Data gathered during the goal-driven mode are only used to refine and prune down the list of impressions.

The application by Dayupay is a subsystem of the GetBetter system developed by \citeA{Azcarraga:2014}.
It is a cloud-based database and web application that stores patient records for viewing and diagnosis by physicians.
It connects patients and physicians who are separated geographically using the web.
Nurses, midwives, and health workers gather information regarding the patient's chief complaint and input details into the system.
The HPI system directs the interviewer when eliciting information from the patient, with the goal of enumerating possible disease impressions.
The symptoms and impressions are sent over the web to the physician who will provide the final diagnosis.    

\section{Research Objectives}
\label{sec:researchobjectives}

\subsection{General Objective}
\label{sec:generalobjective}

To develop a conceptual framework for a highly interactive medical diagnosis question and answer session.

\subsection{Specific Objectives}
\label{sec:specificobjectives}

\begin{enumerate}
    \item To survey the most prevalent medical diseases in Philippine rural areas along with their symptoms and related features;
    %\item To put together a knowledge-base 
    \item To conceive metrics for evaluating the system other than accuracy;
    \item To build a data-driven and goal-driven rule-based system that can simulate question and answer sessions between attending physician and patient;
    \item To build an interface for the designed system that can visualize the inference mechanism; % add
    \item To evaluate the performance of the expert system based on the conceived metrics.
\end{enumerate}


\section{Scope and Limitations of the Research}
\label{sec:scopelimitations}

The research will focus on common medical diseases found in Philippine rural areas. 
These are based on epidemiological studies conducted by the Department of Health of the Philippines.

The expert system will be developed on the Android platform. 
Case records in the GetBetter database will be used to build the expert system model. 

\section{Significance of the Research}
\label{sec:significance}

Barangay health workers serve as a safety net in the monitoring of health in their community and act as outreach for medical staff and midwives.
They play a critical role for vulnerable populations who are isolated geographically or nautically \cite{DirectRelief:2014}.

Visualization of the dynamic interplay between data-driven and goal-driven methods would provide context to the current state of the system.
This would enable health workers using the system to easily follow the logic that the expert system is pursuing. 
As health workers use the system, they would also be more proficient in gathering information and be more knowledgeable in diagnosing diseases.
