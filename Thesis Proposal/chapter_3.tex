%%%%%%%%%%%%%%%%%%%%%%%%%%%%%%%%%%%%%%%%%%%%%%%%%%%%%%%%%%%%%%%%%%%%%%%%%%%%%%%%%%%%%%%%%%%%%%%%%%%%%%
%
%   Filename    : chapter_3.tex 
%
%   Description : This file will contain your Research Methodology.
%                 
%%%%%%%%%%%%%%%%%%%%%%%%%%%%%%%%%%%%%%%%%%%%%%%%%%%%%%%%%%%%%%%%%%%%%%%%%%%%%%%%%%%%%%%%%%%%%%%%%%%%%%

\chapter{Research Methodology}
This chapter lists and discusses the specific steps and activities that will be performed by the proponent to accomplish the project. 
The discussion covers the activities from pre-proposal to Final Thesis Writing.  It also includes an initial discussion on the theoretical
framework to be followed.

\section{Research Concept Formation}
During this stage, the researcher will explore possible topics that will serve as the concept of the research. 
Meetings between the researcher and the research adviser will be held on a regular basis. 
Possible topics will be problems that have an impact on both Computer Science and society.
The researcher will also explore possible tools and solutions that can solve said problems.
Journals and papers of interest will also be examined.

\section{Review of Related Literature}
In order to fully understand the problem, the researcher will assess previous works related to the defined topic.
The researcher will read on studies relevant to medical diagnosis, question-and-answer systems, and rule-based systems.
Approaches used by similar studies will be compared with each other and be considered for usage in the research work.

\section{Data Collection}
Professionals in the medical field will be consulted regarding common diseases and their symptoms.
Academic journals, books, and documents published by medical organizations will also be assessed.
The researcher will also collect patient data available in existing databases.
The information gathered at this stage will serve as the foundation of the system's knowledge base.

During this stage, metrics for evaluating the system will be conceived.
One of the metrics will be focused on the number of steps a session requires to narrow down the diagnosis to a list of plausible illnesses.
The session will also be assessed on the types of questions and complexity of symptoms it asks as the session progresses.
Finally, the system will be evaluated subjectively by medical professionals regarding how it closely simulates an actual question-and-answer session.

\section{System Development}
The system will be divided into modules and will be built in different stages. 

\subsection{Inference Engine}
This module will serve as the decision maker in each question and answer session.
The engine will not necessarily provide a final diagnosis.
It can also yield a list of impressions and plausible illnesses if diseases cannot be discriminated via symptoms alone, such as when laboratory tests are required.
The inference engine will be developed on Java 1.8.

\subsection{Visual Interface}
The interface will display the state of the inference engine at each stage of the session.
Design principles will be utilized in order for users to comprehend the visualizations on display.
The aim is for users - including attending physicians and health workers - to easily visualize the system's decision making process.
Transparency of the system's logic will enable users to build confidence on the system.
This will also assist in correcting errors as the system continues to learn new concepts, diseases, and symptoms.
Visualization will be developed on the Android operating system using the Android SDK.

\subsection{Evaluation}
In this phase, the developed system will be assessed based on the conceived metrics.
 
\section{Documentation}
The researcher will document the progress and results of the study throughout the duration of the study. 

\section{Calendar of Activities}

Table \ref{tab:timetableactivities} shows the schedule of activities for the year 2016.  
Each bullet represents approximately one week worth of activity.

%
%  the following commands will be used for filling up the bullets in the Gantt chart
%
\newcommand{\weekone}{\textbullet}
\newcommand{\weektwo}{\textbullet \textbullet}
\newcommand{\weekthree}{\textbullet \textbullet \textbullet}
\newcommand{\weekfour}{\textbullet \textbullet \textbullet \textbullet}

%
%  alternative to bullet is a star 
%
\begin{comment}
   \newcommand{\weekone}{$\star$}
   \newcommand{\weektwo}{$\star \star$}
   \newcommand{\weekthree}{$\star \star \star$}
   \newcommand{\weekfour}{$\star \star \star \star$ }
\end{comment}



\begin{table}[ht]   %t means place on top, replace with b if you want to place at the bottom
\centering
\caption{Timetable of Activities} \vspace{0.25em}
\begin{tabular}{|p{2in}|c|c|c|c|c|c|c|c|} \hline
\centering Activities (2009) & Jan   & Feb & Mar & Apr & May & Jun & Jul \\ \hline
Study on Prerequisite Knowledge      &   &  & ~~~\weektwo & \weekfour &  &  &  \\ \hline
Review of Existing Racing Strategies & ~~~\weektwo  & \weekfour & \weekfour & \weekfour &  &  &  \\ \hline
Identification of Best Features      &   &  &  & \weekfour & \weektwo~~~ &  &  \\ \hline
Development of Racing Strategies     &   &  &  & ~~~\weektwo & \weekfour & \weektwo~~~ &  \\ \hline
Simulation of Racing Strategies      &   &  &  & ~~~\weektwo & \weekfour & \weekthree~~ &  \\ \hline
Analysis and Interpretation of the Results &   &  &  &  & \weekfour & \weekfour & \weekone~~~~~ \\ \hline
Documentation & ~~~\weektwo  & \weekfour & \weekfour & \weekfour & \weekfour & \weekfour & \weektwo~~~ \\ \hline
\end{tabular}
\label{tab:timetableactivities}
\end{table}

